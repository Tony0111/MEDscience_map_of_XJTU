\documentclass[11pt]{article}

    \usepackage{ctex}
    % 如果上面那行加了还不行,考虑加上显式字体设置(不带sffamily和ttfamily,只设置mainfont)
    \ctexset{
        % 主字体,用于正文
        mainfont = 'Source Han Serif SC', % 例如:思源宋体
        % 无衬线字体,用于标题、粗体等
        sansfont = 'Source Han Sans SC', % 例如:思源黑体
        % 等宽字体,用于代码
        monofont = 'Consolas' % 或 'Consolas', 'Courier New' 等
    }
    \usepackage[breakable]{tcolorbox}
    \usepackage{parskip} % Stop auto-indenting (to mimic markdown behaviour)
    

    % Basic figure setup, for now with no caption control since it's done
    % automatically by Pandoc (which extracts ![](path) syntax from Markdown).
    \usepackage{graphicx}
    % Keep aspect ratio if custom image width or height is specified
    \setkeys{Gin}{keepaspectratio}
    % Maintain compatibility with old templates. Remove in nbconvert 6.0
    \let\Oldincludegraphics\includegraphics
    % Ensure that by default, figures have no caption (until we provide a
    % proper Figure object with a Caption API and a way to capture that
    % in the conversion process - todo).
    \usepackage{caption}
    \DeclareCaptionFormat{nocaption}{}
    \captionsetup{format=nocaption,aboveskip=0pt,belowskip=0pt}

    \usepackage{float}
    \floatplacement{figure}{H} % forces figures to be placed at the correct location
    \usepackage{xcolor} % Allow colors to be defined
    \usepackage{enumerate} % Needed for markdown enumerations to work
    \usepackage{geometry} % Used to adjust the document margins
    \usepackage{amsmath} % Equations
    \usepackage{amssymb} % Equations
    \usepackage{textcomp} % defines textquotesingle
    % Hack from http://tex.stackexchange.com/a/47451/13684:
    \AtBeginDocument{%
        \def\PYZsq{\textquotesingle}% Upright quotes in Pygmentized code
    }
    \usepackage{upquote} % Upright quotes for verbatim code
    \usepackage{eurosym} % defines \euro

    \usepackage{iftex}
    \ifPDFTeX
        \usepackage[T1]{fontenc}
        \IfFileExists{alphabeta.sty}{
              \usepackage{alphabeta}
          }{
              \usepackage[mathletters]{ucs}
              \usepackage[utf8x]{inputenc}
          }
    \else
        \usepackage{fontspec}
        \usepackage{unicode-math}
    \fi

    \usepackage{fancyvrb} % verbatim replacement that allows latex
    \usepackage{grffile} % extends the file name processing of package graphics
                         % to support a larger range
    \makeatletter % fix for old versions of grffile with XeLaTeX
    \@ifpackagelater{grffile}{2019/11/01}
    {
      % Do nothing on new versions
    }
    {
      \def\Gread@@xetex#1{%
        \IfFileExists{"\Gin@base".bb}%
        {\Gread@eps{\Gin@base.bb}}%
        {\Gread@@xetex@aux#1}%
      }
    }
    \makeatother
    \usepackage[Export]{adjustbox} % Used to constrain images to a maximum size
    \adjustboxset{max size={0.9\linewidth}{0.9\paperheight}}

    % The hyperref package gives us a pdf with properly built
    % internal navigation ('pdf bookmarks' for the table of contents,
    % internal cross-reference links, web links for URLs, etc.)
    \usepackage{hyperref}
    % The default LaTeX title has an obnoxious amount of whitespace. By default,
    % titling removes some of it. It also provides customization options.
    \usepackage{titling}
    \usepackage{longtable} % longtable support required by pandoc >1.10
    \usepackage{booktabs}  % table support for pandoc > 1.12.2
    \usepackage{array}     % table support for pandoc >= 2.11.3
    \usepackage{calc}      % table minipage width calculation for pandoc >= 2.11.1
    \usepackage[inline]{enumitem} % IRkernel/repr support (it uses the enumerate* environment)
    \usepackage[normalem]{ulem} % ulem is needed to support strikethroughs (\sout)
                                % normalem makes italics be italics, not underlines
    \usepackage{soul}      % strikethrough (\st) support for pandoc >= 3.0.0
    \usepackage{mathrsfs}
    

    
    % Colors for the hyperref package
    \definecolor{urlcolor}{rgb}{0,.145,.698}
    \definecolor{linkcolor}{rgb}{.71,0.21,0.01}
    \definecolor{citecolor}{rgb}{.12,.54,.11}

    % ANSI colors
    \definecolor{ansi-black}{HTML}{3E424D}
    \definecolor{ansi-black-intense}{HTML}{282C36}
    \definecolor{ansi-red}{HTML}{E75C58}
    \definecolor{ansi-red-intense}{HTML}{B22B31}
    \definecolor{ansi-green}{HTML}{00A250}
    \definecolor{ansi-green-intense}{HTML}{007427}
    \definecolor{ansi-yellow}{HTML}{DDB62B}
    \definecolor{ansi-yellow-intense}{HTML}{B27D12}
    \definecolor{ansi-blue}{HTML}{208FFB}
    \definecolor{ansi-blue-intense}{HTML}{0065CA}
    \definecolor{ansi-magenta}{HTML}{D160C4}
    \definecolor{ansi-magenta-intense}{HTML}{A03196}
    \definecolor{ansi-cyan}{HTML}{60C6C8}
    \definecolor{ansi-cyan-intense}{HTML}{258F8F}
    \definecolor{ansi-white}{HTML}{C5C1B4}
    \definecolor{ansi-white-intense}{HTML}{A1A6B2}
    \definecolor{ansi-default-inverse-fg}{HTML}{FFFFFF}
    \definecolor{ansi-default-inverse-bg}{HTML}{000000}

    % common color for the border for error outputs.
    \definecolor{outerrorbackground}{HTML}{FFDFDF}

    % commands and environments needed by pandoc snippets
    % extracted from the output of `pandoc -s`
    \providecommand{\tightlist}{%
      \setlength{\itemsep}{0pt}\setlength{\parskip}{0pt}}
    \DefineVerbatimEnvironment{Highlighting}{Verbatim}{commandchars=\\\{\}}
    % Add ',fontsize=\small' for more characters per line
    \newenvironment{Shaded}{}{}
    \newcommand{\KeywordTok}[1]{\textcolor[rgb]{0.00,0.44,0.13}{\textbf{{#1}}}}
    \newcommand{\DataTypeTok}[1]{\textcolor[rgb]{0.56,0.13,0.00}{{#1}}}
    \newcommand{\DecValTok}[1]{\textcolor[rgb]{0.25,0.63,0.44}{{#1}}}
    \newcommand{\BaseNTok}[1]{\textcolor[rgb]{0.25,0.63,0.44}{{#1}}}
    \newcommand{\FloatTok}[1]{\textcolor[rgb]{0.25,0.63,0.44}{{#1}}}
    \newcommand{\CharTok}[1]{\textcolor[rgb]{0.25,0.44,0.63}{{#1}}}
    \newcommand{\StringTok}[1]{\textcolor[rgb]{0.25,0.44,0.63}{{#1}}}
    \newcommand{\CommentTok}[1]{\textcolor[rgb]{0.38,0.63,0.69}{\textit{{#1}}}}
    \newcommand{\OtherTok}[1]{\textcolor[rgb]{0.00,0.44,0.13}{{#1}}}
    \newcommand{\AlertTok}[1]{\textcolor[rgb]{1.00,0.00,0.00}{\textbf{{#1}}}}
    \newcommand{\FunctionTok}[1]{\textcolor[rgb]{0.02,0.16,0.49}{{#1}}}
    \newcommand{\RegionMarkerTok}[1]{{#1}}
    \newcommand{\ErrorTok}[1]{\textcolor[rgb]{1.00,0.00,0.00}{\textbf{{#1}}}}
    \newcommand{\NormalTok}[1]{{#1}}

    % Additional commands for more recent versions of Pandoc
    \newcommand{\ConstantTok}[1]{\textcolor[rgb]{0.53,0.00,0.00}{{#1}}}
    \newcommand{\SpecialCharTok}[1]{\textcolor[rgb]{0.25,0.44,0.63}{{#1}}}
    \newcommand{\VerbatimStringTok}[1]{\textcolor[rgb]{0.25,0.44,0.63}{{#1}}}
    \newcommand{\SpecialStringTok}[1]{\textcolor[rgb]{0.73,0.40,0.53}{{#1}}}
    \newcommand{\ImportTok}[1]{{#1}}
    \newcommand{\DocumentationTok}[1]{\textcolor[rgb]{0.73,0.13,0.13}{\textit{{#1}}}}
    \newcommand{\AnnotationTok}[1]{\textcolor[rgb]{0.38,0.63,0.69}{\textbf{\textit{{#1}}}}}
    \newcommand{\CommentVarTok}[1]{\textcolor[rgb]{0.38,0.63,0.69}{\textbf{\textit{{#1}}}}}
    \newcommand{\VariableTok}[1]{\textcolor[rgb]{0.10,0.09,0.49}{{#1}}}
    \newcommand{\ControlFlowTok}[1]{\textcolor[rgb]{0.00,0.44,0.13}{\textbf{{#1}}}}
    \newcommand{\OperatorTok}[1]{\textcolor[rgb]{0.40,0.40,0.40}{{#1}}}
    \newcommand{\BuiltInTok}[1]{{#1}}
    \newcommand{\ExtensionTok}[1]{{#1}}
    \newcommand{\PreprocessorTok}[1]{\textcolor[rgb]{0.74,0.48,0.00}{{#1}}}
    \newcommand{\AttributeTok}[1]{\textcolor[rgb]{0.49,0.56,0.16}{{#1}}}
    \newcommand{\InformationTok}[1]{\textcolor[rgb]{0.38,0.63,0.69}{\textbf{\textit{{#1}}}}}
    \newcommand{\WarningTok}[1]{\textcolor[rgb]{0.38,0.63,0.69}{\textbf{\textit{{#1}}}}}


    % Define a nice break command that doesn't care if a line doesn't already
    % exist.
    \def\br{\hspace*{\fill} \\* }
    % Math Jax compatibility definitions
    \def\gt{>}
    \def\lt{<}
    \let\Oldtex\TeX
    \let\Oldlatex\LaTeX
    \renewcommand{\TeX}{\textrm{\Oldtex}}
    \renewcommand{\LaTeX}{\textrm{\Oldlatex}}
    % Document parameters
    % Document title
    \title{data\_combine}
    
    
    
    
    
    
    
% Pygments definitions
\makeatletter
\def\PY@reset{\let\PY@it=\relax \let\PY@bf=\relax%
    \let\PY@ul=\relax \let\PY@tc=\relax%
    \let\PY@bc=\relax \let\PY@ff=\relax}
\def\PY@tok#1{\csname PY@tok@#1\endcsname}
\def\PY@toks#1+{\ifx\relax#1\empty\else%
    \PY@tok{#1}\expandafter\PY@toks\fi}
\def\PY@do#1{\PY@bc{\PY@tc{\PY@ul{%
    \PY@it{\PY@bf{\PY@ff{#1}}}}}}}
\def\PY#1#2{\PY@reset\PY@toks#1+\relax+\PY@do{#2}}

\@namedef{PY@tok@w}{\def\PY@tc##1{\textcolor[rgb]{0.73,0.73,0.73}{##1}}}
\@namedef{PY@tok@c}{\let\PY@it=\textit\def\PY@tc##1{\textcolor[rgb]{0.24,0.48,0.48}{##1}}}
\@namedef{PY@tok@cp}{\def\PY@tc##1{\textcolor[rgb]{0.61,0.40,0.00}{##1}}}
\@namedef{PY@tok@k}{\let\PY@bf=\textbf\def\PY@tc##1{\textcolor[rgb]{0.00,0.50,0.00}{##1}}}
\@namedef{PY@tok@kp}{\def\PY@tc##1{\textcolor[rgb]{0.00,0.50,0.00}{##1}}}
\@namedef{PY@tok@kt}{\def\PY@tc##1{\textcolor[rgb]{0.69,0.00,0.25}{##1}}}
\@namedef{PY@tok@o}{\def\PY@tc##1{\textcolor[rgb]{0.40,0.40,0.40}{##1}}}
\@namedef{PY@tok@ow}{\let\PY@bf=\textbf\def\PY@tc##1{\textcolor[rgb]{0.67,0.13,1.00}{##1}}}
\@namedef{PY@tok@nb}{\def\PY@tc##1{\textcolor[rgb]{0.00,0.50,0.00}{##1}}}
\@namedef{PY@tok@nf}{\def\PY@tc##1{\textcolor[rgb]{0.00,0.00,1.00}{##1}}}
\@namedef{PY@tok@nc}{\let\PY@bf=\textbf\def\PY@tc##1{\textcolor[rgb]{0.00,0.00,1.00}{##1}}}
\@namedef{PY@tok@nn}{\let\PY@bf=\textbf\def\PY@tc##1{\textcolor[rgb]{0.00,0.00,1.00}{##1}}}
\@namedef{PY@tok@ne}{\let\PY@bf=\textbf\def\PY@tc##1{\textcolor[rgb]{0.80,0.25,0.22}{##1}}}
\@namedef{PY@tok@nv}{\def\PY@tc##1{\textcolor[rgb]{0.10,0.09,0.49}{##1}}}
\@namedef{PY@tok@no}{\def\PY@tc##1{\textcolor[rgb]{0.53,0.00,0.00}{##1}}}
\@namedef{PY@tok@nl}{\def\PY@tc##1{\textcolor[rgb]{0.46,0.46,0.00}{##1}}}
\@namedef{PY@tok@ni}{\let\PY@bf=\textbf\def\PY@tc##1{\textcolor[rgb]{0.44,0.44,0.44}{##1}}}
\@namedef{PY@tok@na}{\def\PY@tc##1{\textcolor[rgb]{0.41,0.47,0.13}{##1}}}
\@namedef{PY@tok@nt}{\let\PY@bf=\textbf\def\PY@tc##1{\textcolor[rgb]{0.00,0.50,0.00}{##1}}}
\@namedef{PY@tok@nd}{\def\PY@tc##1{\textcolor[rgb]{0.67,0.13,1.00}{##1}}}
\@namedef{PY@tok@s}{\def\PY@tc##1{\textcolor[rgb]{0.73,0.13,0.13}{##1}}}
\@namedef{PY@tok@sd}{\let\PY@it=\textit\def\PY@tc##1{\textcolor[rgb]{0.73,0.13,0.13}{##1}}}
\@namedef{PY@tok@si}{\let\PY@bf=\textbf\def\PY@tc##1{\textcolor[rgb]{0.64,0.35,0.47}{##1}}}
\@namedef{PY@tok@se}{\let\PY@bf=\textbf\def\PY@tc##1{\textcolor[rgb]{0.67,0.36,0.12}{##1}}}
\@namedef{PY@tok@sr}{\def\PY@tc##1{\textcolor[rgb]{0.64,0.35,0.47}{##1}}}
\@namedef{PY@tok@ss}{\def\PY@tc##1{\textcolor[rgb]{0.10,0.09,0.49}{##1}}}
\@namedef{PY@tok@sx}{\def\PY@tc##1{\textcolor[rgb]{0.00,0.50,0.00}{##1}}}
\@namedef{PY@tok@m}{\def\PY@tc##1{\textcolor[rgb]{0.40,0.40,0.40}{##1}}}
\@namedef{PY@tok@gh}{\let\PY@bf=\textbf\def\PY@tc##1{\textcolor[rgb]{0.00,0.00,0.50}{##1}}}
\@namedef{PY@tok@gu}{\let\PY@bf=\textbf\def\PY@tc##1{\textcolor[rgb]{0.50,0.00,0.50}{##1}}}
\@namedef{PY@tok@gd}{\def\PY@tc##1{\textcolor[rgb]{0.63,0.00,0.00}{##1}}}
\@namedef{PY@tok@gi}{\def\PY@tc##1{\textcolor[rgb]{0.00,0.52,0.00}{##1}}}
\@namedef{PY@tok@gr}{\def\PY@tc##1{\textcolor[rgb]{0.89,0.00,0.00}{##1}}}
\@namedef{PY@tok@ge}{\let\PY@it=\textit}
\@namedef{PY@tok@gs}{\let\PY@bf=\textbf}
\@namedef{PY@tok@ges}{\let\PY@bf=\textbf\let\PY@it=\textit}
\@namedef{PY@tok@gp}{\let\PY@bf=\textbf\def\PY@tc##1{\textcolor[rgb]{0.00,0.00,0.50}{##1}}}
\@namedef{PY@tok@go}{\def\PY@tc##1{\textcolor[rgb]{0.44,0.44,0.44}{##1}}}
\@namedef{PY@tok@gt}{\def\PY@tc##1{\textcolor[rgb]{0.00,0.27,0.87}{##1}}}
\@namedef{PY@tok@err}{\def\PY@bc##1{{\setlength{\fboxsep}{\string -\fboxrule}\fcolorbox[rgb]{1.00,0.00,0.00}{1,1,1}{\strut ##1}}}}
\@namedef{PY@tok@kc}{\let\PY@bf=\textbf\def\PY@tc##1{\textcolor[rgb]{0.00,0.50,0.00}{##1}}}
\@namedef{PY@tok@kd}{\let\PY@bf=\textbf\def\PY@tc##1{\textcolor[rgb]{0.00,0.50,0.00}{##1}}}
\@namedef{PY@tok@kn}{\let\PY@bf=\textbf\def\PY@tc##1{\textcolor[rgb]{0.00,0.50,0.00}{##1}}}
\@namedef{PY@tok@kr}{\let\PY@bf=\textbf\def\PY@tc##1{\textcolor[rgb]{0.00,0.50,0.00}{##1}}}
\@namedef{PY@tok@bp}{\def\PY@tc##1{\textcolor[rgb]{0.00,0.50,0.00}{##1}}}
\@namedef{PY@tok@fm}{\def\PY@tc##1{\textcolor[rgb]{0.00,0.00,1.00}{##1}}}
\@namedef{PY@tok@vc}{\def\PY@tc##1{\textcolor[rgb]{0.10,0.09,0.49}{##1}}}
\@namedef{PY@tok@vg}{\def\PY@tc##1{\textcolor[rgb]{0.10,0.09,0.49}{##1}}}
\@namedef{PY@tok@vi}{\def\PY@tc##1{\textcolor[rgb]{0.10,0.09,0.49}{##1}}}
\@namedef{PY@tok@vm}{\def\PY@tc##1{\textcolor[rgb]{0.10,0.09,0.49}{##1}}}
\@namedef{PY@tok@sa}{\def\PY@tc##1{\textcolor[rgb]{0.73,0.13,0.13}{##1}}}
\@namedef{PY@tok@sb}{\def\PY@tc##1{\textcolor[rgb]{0.73,0.13,0.13}{##1}}}
\@namedef{PY@tok@sc}{\def\PY@tc##1{\textcolor[rgb]{0.73,0.13,0.13}{##1}}}
\@namedef{PY@tok@dl}{\def\PY@tc##1{\textcolor[rgb]{0.73,0.13,0.13}{##1}}}
\@namedef{PY@tok@s2}{\def\PY@tc##1{\textcolor[rgb]{0.73,0.13,0.13}{##1}}}
\@namedef{PY@tok@sh}{\def\PY@tc##1{\textcolor[rgb]{0.73,0.13,0.13}{##1}}}
\@namedef{PY@tok@s1}{\def\PY@tc##1{\textcolor[rgb]{0.73,0.13,0.13}{##1}}}
\@namedef{PY@tok@mb}{\def\PY@tc##1{\textcolor[rgb]{0.40,0.40,0.40}{##1}}}
\@namedef{PY@tok@mf}{\def\PY@tc##1{\textcolor[rgb]{0.40,0.40,0.40}{##1}}}
\@namedef{PY@tok@mh}{\def\PY@tc##1{\textcolor[rgb]{0.40,0.40,0.40}{##1}}}
\@namedef{PY@tok@mi}{\def\PY@tc##1{\textcolor[rgb]{0.40,0.40,0.40}{##1}}}
\@namedef{PY@tok@il}{\def\PY@tc##1{\textcolor[rgb]{0.40,0.40,0.40}{##1}}}
\@namedef{PY@tok@mo}{\def\PY@tc##1{\textcolor[rgb]{0.40,0.40,0.40}{##1}}}
\@namedef{PY@tok@ch}{\let\PY@it=\textit\def\PY@tc##1{\textcolor[rgb]{0.24,0.48,0.48}{##1}}}
\@namedef{PY@tok@cm}{\let\PY@it=\textit\def\PY@tc##1{\textcolor[rgb]{0.24,0.48,0.48}{##1}}}
\@namedef{PY@tok@cpf}{\let\PY@it=\textit\def\PY@tc##1{\textcolor[rgb]{0.24,0.48,0.48}{##1}}}
\@namedef{PY@tok@c1}{\let\PY@it=\textit\def\PY@tc##1{\textcolor[rgb]{0.24,0.48,0.48}{##1}}}
\@namedef{PY@tok@cs}{\let\PY@it=\textit\def\PY@tc##1{\textcolor[rgb]{0.24,0.48,0.48}{##1}}}

\def\PYZbs{\char`\\}
\def\PYZus{\char`\_}
\def\PYZob{\char`\{}
\def\PYZcb{\char`\}}
\def\PYZca{\char`\^}
\def\PYZam{\char`\&}
\def\PYZlt{\char`\<}
\def\PYZgt{\char`\>}
\def\PYZsh{\char`\#}
\def\PYZpc{\char`\%}
\def\PYZdl{\char`\$}
\def\PYZhy{\char`\-}
\def\PYZsq{\char`\'}
\def\PYZdq{\char`\"}
\def\PYZti{\char`\~}
% for compatibility with earlier versions
\def\PYZat{@}
\def\PYZlb{[}
\def\PYZrb{]}
\makeatother


    % For linebreaks inside Verbatim environment from package fancyvrb.
    \makeatletter
        \newbox\Wrappedcontinuationbox
        \newbox\Wrappedvisiblespacebox
        \newcommand*\Wrappedvisiblespace {\textcolor{red}{\textvisiblespace}}
        \newcommand*\Wrappedcontinuationsymbol {\textcolor{red}{\llap{\tiny$\m@th\hookrightarrow$}}}
        \newcommand*\Wrappedcontinuationindent {3ex }
        \newcommand*\Wrappedafterbreak {\kern\Wrappedcontinuationindent\copy\Wrappedcontinuationbox}
        % Take advantage of the already applied Pygments mark-up to insert
        % potential linebreaks for TeX processing.
        %        {, <, #, %, $, ' and ": go to next line.
        %        _, }, ^, &, >, - and ~: stay at end of broken line.
        % Use of \textquotesingle for straight quote.
        \newcommand*\Wrappedbreaksatspecials {%
            \def\PYGZus{\discretionary{\char`\_}{\Wrappedafterbreak}{\char`\_}}%
            \def\PYGZob{\discretionary{}{\Wrappedafterbreak\char`\{}{\char`\{}}%
            \def\PYGZcb{\discretionary{\char`\}}{\Wrappedafterbreak}{\char`\}}}%
            \def\PYGZca{\discretionary{\char`\^}{\Wrappedafterbreak}{\char`\^}}%
            \def\PYGZam{\discretionary{\char`\&}{\Wrappedafterbreak}{\char`\&}}%
            \def\PYGZlt{\discretionary{}{\Wrappedafterbreak\char`\<}{\char`\<}}%
            \def\PYGZgt{\discretionary{\char`\>}{\Wrappedafterbreak}{\char`\>}}%
            \def\PYGZsh{\discretionary{}{\Wrappedafterbreak\char`\#}{\char`\#}}%
            \def\PYGZpc{\discretionary{}{\Wrappedafterbreak\char`\%}{\char`\%}}%
            \def\PYGZdl{\discretionary{}{\Wrappedafterbreak\char`\$}{\char`\$}}%
            \def\PYGZhy{\discretionary{\char`\-}{\Wrappedafterbreak}{\char`\-}}%
            \def\PYGZsq{\discretionary{}{\Wrappedafterbreak\textquotesingle}{\textquotesingle}}%
            \def\PYGZdq{\discretionary{}{\Wrappedafterbreak\char`\"}{\char`\"}}%
            \def\PYGZti{\discretionary{\char`\~}{\Wrappedafterbreak}{\char`\~}}%
        }
        % Some characters . , ; ? ! / are not pygmentized.
        % This macro makes them "active" and they will insert potential linebreaks
        \newcommand*\Wrappedbreaksatpunct {%
            \lccode`\~`\.\lowercase{\def~}{\discretionary{\hbox{\char`\.}}{\Wrappedafterbreak}{\hbox{\char`\.}}}%
            \lccode`\~`\,\lowercase{\def~}{\discretionary{\hbox{\char`\,}}{\Wrappedafterbreak}{\hbox{\char`\,}}}%
            \lccode`\~`\;\lowercase{\def~}{\discretionary{\hbox{\char`\;}}{\Wrappedafterbreak}{\hbox{\char`\;}}}%
            \lccode`\~`\:\lowercase{\def~}{\discretionary{\hbox{\char`\:}}{\Wrappedafterbreak}{\hbox{\char`\:}}}%
            \lccode`\~`\?\lowercase{\def~}{\discretionary{\hbox{\char`\?}}{\Wrappedafterbreak}{\hbox{\char`\?}}}%
            \lccode`\~`\!\lowercase{\def~}{\discretionary{\hbox{\char`\!}}{\Wrappedafterbreak}{\hbox{\char`\!}}}%
            \lccode`\~`\/\lowercase{\def~}{\discretionary{\hbox{\char`\/}}{\Wrappedafterbreak}{\hbox{\char`\/}}}%
            \catcode`\.\active
            \catcode`\,\active
            \catcode`\;\active
            \catcode`\:\active
            \catcode`\?\active
            \catcode`\!\active
            \catcode`\/\active
            \lccode`\~`\~
        }
    \makeatother

    \let\OriginalVerbatim=\Verbatim
    \makeatletter
    \renewcommand{\Verbatim}[1][1]{%
        %\parskip\z@skip
        \sbox\Wrappedcontinuationbox {\Wrappedcontinuationsymbol}%
        \sbox\Wrappedvisiblespacebox {\FV@SetupFont\Wrappedvisiblespace}%
        \def\FancyVerbFormatLine ##1{\hsize\linewidth
            \vtop{\raggedright\hyphenpenalty\z@\exhyphenpenalty\z@
                \doublehyphendemerits\z@\finalhyphendemerits\z@
                \strut ##1\strut}%
        }%
        % If the linebreak is at a space, the latter will be displayed as visible
        % space at end of first line, and a continuation symbol starts next line.
        % Stretch/shrink are however usually zero for typewriter font.
        \def\FV@Space {%
            \nobreak\hskip\z@ plus\fontdimen3\font minus\fontdimen4\font
            \discretionary{\copy\Wrappedvisiblespacebox}{\Wrappedafterbreak}
            {\kern\fontdimen2\font}%
        }%

        % Allow breaks at special characters using \PYG... macros.
        \Wrappedbreaksatspecials
        % Breaks at punctuation characters . , ; ? ! and / need catcode=\active
        \OriginalVerbatim[#1,codes*=\Wrappedbreaksatpunct]%
    }
    \makeatother

    % Exact colors from NB
    \definecolor{incolor}{HTML}{303F9F}
    \definecolor{outcolor}{HTML}{D84315}
    \definecolor{cellborder}{HTML}{CFCFCF}
    \definecolor{cellbackground}{HTML}{F7F7F7}

    % prompt
    \makeatletter
    \newcommand{\boxspacing}{\kern\kvtcb@left@rule\kern\kvtcb@boxsep}
    \makeatother
    \newcommand{\prompt}[4]{
        {\ttfamily\llap{{\color{#2}[#3]:\hspace{3pt}#4}}\vspace{-\baselineskip}}
    }
    

    
    % Prevent overflowing lines due to hard-to-break entities
    \sloppy
    % Setup hyperref package
    \hypersetup{
      breaklinks=true,  % so long urls are correctly broken across lines
      colorlinks=true,
      urlcolor=urlcolor,
      linkcolor=linkcolor,
      citecolor=citecolor,
      }
    % Slightly bigger margins than the latex defaults
    
    \geometry{verbose,tmargin=1in,bmargin=1in,lmargin=1in,rmargin=1in}
    
    

\begin{document}
    
    \title{Data Combine on Python}
    \author{\_达\_\_} % 作者名字,注意下划线要转义
    \date{\today}
    \maketitle
    
    

    
    \section{文档介绍:}\label{ux6587ux6863ux4ecbux7ecd}

\begin{itemize}
\tightlist
\item
  从zotero导出的医学数据csv文件一共有五个(按照年份)
\item
  数据清洗时,合并为一个文件更便于处理,这就是本文件所做的事情
\item
  这个文档处理完后的数据会到\texttt{data\_process.ipynb}中继续后续处理
\end{itemize}

    \begin{tcolorbox}[breakable, size=fbox, boxrule=1pt, pad at break*=1mm,colback=cellbackground, colframe=cellborder]
\prompt{In}{incolor}{6}{\boxspacing}
\begin{Verbatim}[commandchars=\\\{\}]
\PY{k+kn}{import}\PY{+w}{ }\PY{n+nn}{pandas}\PY{+w}{ }\PY{k}{as}\PY{+w}{ }\PY{n+nn}{pd}
\PY{k+kn}{import}\PY{+w}{ }\PY{n+nn}{glob}
\PY{k+kn}{import}\PY{+w}{ }\PY{n+nn}{os}
\end{Verbatim}
\end{tcolorbox}

    \subsection{定义文件路径}\label{ux5b9aux4e49ux6587ux4ef6ux8defux5f84}

\begin{itemize}
\tightlist
\item
  意思让电脑知道文件的位置
\end{itemize}

    \begin{tcolorbox}[breakable, size=fbox, boxrule=1pt, pad at break*=1mm,colback=cellbackground, colframe=cellborder]
\prompt{In}{incolor}{7}{\boxspacing}
\begin{Verbatim}[commandchars=\\\{\}]
\PY{c+c1}{\PYZsh{} \PYZhy{}\PYZhy{}\PYZhy{} 1. 定义文件路径 \PYZhy{}\PYZhy{}\PYZhy{}}
\PY{c+c1}{\PYZsh{} 定义数据文件夹的相对路径}
\PY{c+c1}{\PYZsh{} . 表示当前 Notebook 文件所在的目录}
\PY{c+c1}{\PYZsh{} .. 表示上一级目录}
\PY{c+c1}{\PYZsh{} 假设你的 Notebook 在 \PYZsq{}notebooks/\PYZsq{} 文件夹下,而数据文件夹 \PYZsq{}data/raw/\PYZsq{} }
\PY{c+c1}{\PYZsh{} 在仓库的根目录下,所以需要先回退一级 (..) 再进入 \PYZsq{}data/raw/\PYZsq{}}
\PY{n}{data\PYZus{}path} \PY{o}{=} \PY{n}{os}\PY{o}{.}\PY{n}{path}\PY{o}{.}\PY{n}{join}\PY{p}{(}\PY{n}{os}\PY{o}{.}\PY{n}{getcwd}\PY{p}{(}\PY{p}{)}\PY{p}{,} \PY{l+s+s1}{\PYZsq{}}\PY{l+s+s1}{..}\PY{l+s+s1}{\PYZsq{}}\PY{p}{,} \PY{l+s+s1}{\PYZsq{}}\PY{l+s+s1}{data}\PY{l+s+s1}{\PYZsq{}}\PY{p}{,} \PY{l+s+s1}{\PYZsq{}}\PY{l+s+s1}{raw}\PY{l+s+s1}{\PYZsq{}}\PY{p}{)} 
\PY{n}{processed\PYZus{}data\PYZus{}path} \PY{o}{=} \PY{n}{os}\PY{o}{.}\PY{n}{path}\PY{o}{.}\PY{n}{join}\PY{p}{(}\PY{n}{os}\PY{o}{.}\PY{n}{getcwd}\PY{p}{(}\PY{p}{)}\PY{p}{,} \PY{l+s+s1}{\PYZsq{}}\PY{l+s+s1}{..}\PY{l+s+s1}{\PYZsq{}}\PY{p}{,} \PY{l+s+s1}{\PYZsq{}}\PY{l+s+s1}{data}\PY{l+s+s1}{\PYZsq{}}\PY{p}{,} \PY{l+s+s1}{\PYZsq{}}\PY{l+s+s1}{processed}\PY{l+s+s1}{\PYZsq{}}\PY{p}{)}
\end{Verbatim}
\end{tcolorbox}

    \begin{tcolorbox}[breakable, size=fbox, boxrule=1pt, pad at break*=1mm,colback=cellbackground, colframe=cellborder]
\prompt{In}{incolor}{8}{\boxspacing}
\begin{Verbatim}[commandchars=\\\{\}]
\PY{c+c1}{\PYZsh{} 使用glob找到所有以.csv结尾的文件}
\PY{c+c1}{\PYZsh{} 这个路径是相对于当前 Notebook 文件执行时的工作目录而言的}
\PY{n}{all\PYZus{}csv\PYZus{}files} \PY{o}{=} \PY{n}{glob}\PY{o}{.}\PY{n}{glob}\PY{p}{(}\PY{n}{os}\PY{o}{.}\PY{n}{path}\PY{o}{.}\PY{n}{join}\PY{p}{(}\PY{n}{data\PYZus{}path}\PY{p}{,} \PY{l+s+s2}{\PYZdq{}}\PY{l+s+s2}{*.csv}\PY{l+s+s2}{\PYZdq{}}\PY{p}{)}\PY{p}{)}
\PY{c+c1}{\PYZsh{} 打印出来检查一下,确保文件都被找到了}
\PY{n+nb}{print}\PY{p}{(}\PY{l+s+s2}{\PYZdq{}}\PY{l+s+s2}{以下CSV文件将被读取:}\PY{l+s+s2}{\PYZdq{}}\PY{p}{)}
\PY{k}{if} \PY{o+ow}{not} \PY{n}{all\PYZus{}csv\PYZus{}files}\PY{p}{:}
    \PY{n+nb}{print}\PY{p}{(}\PY{l+s+s2}{\PYZdq{}}\PY{l+s+s2}{\PYZhy{}\PYZhy{}\PYZhy{} 警告:未找到任何CSV文件!请检查:\PYZhy{}\PYZhy{}\PYZhy{}}\PY{l+s+s2}{\PYZdq{}}\PY{p}{)}
    \PY{n+nb}{print}\PY{p}{(}\PY{l+s+sa}{f}\PY{l+s+s2}{\PYZdq{}}\PY{l+s+s2}{1. notebook 的位置是否正确? (当前工作目录: }\PY{l+s+si}{\PYZob{}}\PY{n}{os}\PY{o}{.}\PY{n}{getcwd}\PY{p}{(}\PY{p}{)}\PY{l+s+si}{\PYZcb{}}\PY{l+s+s2}{)}\PY{l+s+s2}{\PYZdq{}}\PY{p}{)}
    \PY{n+nb}{print}\PY{p}{(}\PY{l+s+sa}{f}\PY{l+s+s2}{\PYZdq{}}\PY{l+s+s2}{2. }\PY{l+s+s2}{\PYZsq{}}\PY{l+s+s2}{data/raw/}\PY{l+s+s2}{\PYZsq{}}\PY{l+s+s2}{ 文件夹是否在你的仓库根目录下?}\PY{l+s+s2}{\PYZdq{}}\PY{p}{)}
    \PY{n+nb}{print}\PY{p}{(}\PY{l+s+sa}{f}\PY{l+s+s2}{\PYZdq{}}\PY{l+s+s2}{3. }\PY{l+s+s2}{\PYZsq{}}\PY{l+s+s2}{data/raw/}\PY{l+s+s2}{\PYZsq{}}\PY{l+s+s2}{ 文件夹内是否包含您要读取的 .csv 文件?}\PY{l+s+s2}{\PYZdq{}}\PY{p}{)}
\PY{k}{else}\PY{p}{:}
    \PY{k}{for} \PY{n}{f} \PY{o+ow}{in} \PY{n+nb}{sorted}\PY{p}{(}\PY{n}{all\PYZus{}csv\PYZus{}files}\PY{p}{)}\PY{p}{:} \PY{c+c1}{\PYZsh{} 使用 sorted 保证输出顺序一致}
        \PY{n+nb}{print}\PY{p}{(}\PY{n}{f}\PY{p}{)}
\end{Verbatim}
\end{tcolorbox}

    \begin{Verbatim}[commandchars=\\\{\}]
以下CSV文件将被读取:
d:\textbackslash{}MEDscience\_map\_of\_XJTU\textbackslash{}notebook\textbackslash{}..\textbackslash{}data\textbackslash{}raw\textbackslash{}pubmed-XianJiaoto-set2021.csv
d:\textbackslash{}MEDscience\_map\_of\_XJTU\textbackslash{}notebook\textbackslash{}..\textbackslash{}data\textbackslash{}raw\textbackslash{}pubmed-XianJiaoto-set2022.csv
d:\textbackslash{}MEDscience\_map\_of\_XJTU\textbackslash{}notebook\textbackslash{}..\textbackslash{}data\textbackslash{}raw\textbackslash{}pubmed-XianJiaoto-set2023.csv
d:\textbackslash{}MEDscience\_map\_of\_XJTU\textbackslash{}notebook\textbackslash{}..\textbackslash{}data\textbackslash{}raw\textbackslash{}pubmed-XianJiaoto-set2024.csv
d:\textbackslash{}MEDscience\_map\_of\_XJTU\textbackslash{}notebook\textbackslash{}..\textbackslash{}data\textbackslash{}raw\textbackslash{}pubmed-XianJiaoto-set2025.csv
    \end{Verbatim}

    \subsection{关键步骤:合并操作}\label{ux5173ux952eux6b65ux9aa4ux5408ux5e76ux64cdux4f5c}

    \begin{tcolorbox}[breakable, size=fbox, boxrule=1pt, pad at break*=1mm,colback=cellbackground, colframe=cellborder]
\prompt{In}{incolor}{9}{\boxspacing}
\begin{Verbatim}[commandchars=\\\{\}]
\PY{c+c1}{\PYZsh{} \PYZhy{}\PYZhy{}\PYZhy{} 2. 循环读取并合并 \PYZhy{}\PYZhy{}\PYZhy{}}
\PY{c+c1}{\PYZsh{} 创建一个空列表,用来存放每个文件读取后的DataFrame}
\PY{n}{df\PYZus{}list} \PY{o}{=} \PY{p}{[}\PY{p}{]}

\PY{k}{for} \PY{n}{file} \PY{o+ow}{in} \PY{n}{all\PYZus{}csv\PYZus{}files}\PY{p}{:}
    \PY{c+c1}{\PYZsh{} 循环读取每一个csv文件}
    \PY{n}{df\PYZus{}temp} \PY{o}{=} \PY{n}{pd}\PY{o}{.}\PY{n}{read\PYZus{}csv}\PY{p}{(}\PY{n}{file}\PY{p}{)}
    \PY{c+c1}{\PYZsh{} 将读取的DataFrame添加到列表中}
    \PY{n}{df\PYZus{}list}\PY{o}{.}\PY{n}{append}\PY{p}{(}\PY{n}{df\PYZus{}temp}\PY{p}{)}

\PY{c+c1}{\PYZsh{} 使用pd.concat()函数将列表中的所有DataFrame一次性合并}
\PY{c+c1}{\PYZsh{} ignore\PYZus{}index=True 是非常重要的一步,它会重新生成一套连续的索引,避免索引重复}
\PY{n}{df\PYZus{}combined} \PY{o}{=} \PY{n}{pd}\PY{o}{.}\PY{n}{concat}\PY{p}{(}\PY{n}{df\PYZus{}list}\PY{p}{,} \PY{n}{ignore\PYZus{}index}\PY{o}{=}\PY{k+kc}{True}\PY{p}{)}
\end{Verbatim}
\end{tcolorbox}

    \subsection{保存合并后的文件}\label{ux4fddux5b58ux5408ux5e76ux540eux7684ux6587ux4ef6}

    \begin{tcolorbox}[breakable, size=fbox, boxrule=1pt, pad at break*=1mm,colback=cellbackground, colframe=cellborder]
\prompt{In}{incolor}{12}{\boxspacing}
\begin{Verbatim}[commandchars=\\\{\}]
\PY{c+c1}{\PYZsh{} \PYZhy{}\PYZhy{}\PYZhy{} 3. 保存合并后的文件 \PYZhy{}\PYZhy{}\PYZhy{}}
\PY{c+c1}{\PYZsh{} 定义新文件的保存路径,使用 processed\PYZus{}data\PYZus{}path}
\PY{n}{output\PYZus{}filename} \PY{o}{=} \PY{n}{os}\PY{o}{.}\PY{n}{path}\PY{o}{.}\PY{n}{join}\PY{p}{(}\PY{n}{processed\PYZus{}data\PYZus{}path}\PY{p}{,} \PY{l+s+s1}{\PYZsq{}}\PY{l+s+s1}{zotero\PYZus{}data\PYZus{}combined.csv}\PY{l+s+s1}{\PYZsq{}}\PY{p}{)}
\PY{c+c1}{\PYZsh{} 将合并后的大文件保存起来}
\PY{k}{try}\PY{p}{:}
    \PY{n}{df\PYZus{}combined}\PY{o}{.}\PY{n}{to\PYZus{}csv}\PY{p}{(}\PY{n+nb}{str}\PY{p}{(}\PY{n}{output\PYZus{}filename}\PY{p}{)}\PY{p}{,} \PY{n}{index}\PY{o}{=}\PY{k+kc}{False}\PY{p}{)} \PY{c+c1}{\PYZsh{} to\PYZus{}csv 需要字符串路径}
    \PY{n+nb}{print}\PY{p}{(}\PY{l+s+sa}{f}\PY{l+s+s2}{\PYZdq{}}\PY{l+s+se}{\PYZbs{}n}\PY{l+s+s2}{合并完成!总共包含 }\PY{l+s+si}{\PYZob{}}\PY{n+nb}{len}\PY{p}{(}\PY{n}{df\PYZus{}combined}\PY{p}{)}\PY{l+s+si}{\PYZcb{}}\PY{l+s+s2}{ 条记录。}\PY{l+s+s2}{\PYZdq{}}\PY{p}{)}
    \PY{n+nb}{print}\PY{p}{(}\PY{l+s+sa}{f}\PY{l+s+s2}{\PYZdq{}}\PY{l+s+s2}{数据已保存到: }\PY{l+s+si}{\PYZob{}}\PY{n}{output\PYZus{}filename}\PY{l+s+si}{\PYZcb{}}\PY{l+s+s2}{\PYZdq{}}\PY{p}{)}
\PY{k}{except} \PY{n+ne}{Exception} \PY{k}{as} \PY{n}{e}\PY{p}{:}
    \PY{n+nb}{print}\PY{p}{(}\PY{l+s+sa}{f}\PY{l+s+s2}{\PYZdq{}}\PY{l+s+s2}{保存文件 }\PY{l+s+si}{\PYZob{}}\PY{n}{output\PYZus{}filename}\PY{l+s+si}{\PYZcb{}}\PY{l+s+s2}{ 时出错: }\PY{l+s+si}{\PYZob{}}\PY{n}{e}\PY{l+s+si}{\PYZcb{}}\PY{l+s+s2}{\PYZdq{}}\PY{p}{)}
    
\end{Verbatim}
\end{tcolorbox}

    \begin{Verbatim}[commandchars=\\\{\}]

合并完成!总共包含 25105 条记录。
数据已保存到:
d:\textbackslash{}MEDscience\_map\_of\_XJTU\textbackslash{}notebook\textbackslash{}..\textbackslash{}data\textbackslash{}processed\textbackslash{}zotero\_data\_combined.csv
    \end{Verbatim}

    \subsection{检查合并后的数据}\label{ux68c0ux67e5ux5408ux5e76ux540eux7684ux6570ux636e}

    \begin{tcolorbox}[breakable, size=fbox, boxrule=1pt, pad at break*=1mm,colback=cellbackground, colframe=cellborder]
\prompt{In}{incolor}{13}{\boxspacing}
\begin{Verbatim}[commandchars=\\\{\}]
\PY{c+c1}{\PYZsh{} \PYZhy{}\PYZhy{}\PYZhy{} 4. 检查合并后的数据 \PYZhy{}\PYZhy{}\PYZhy{}}
\PY{c+c1}{\PYZsh{} 显示前5行,快速预览}
\PY{n+nb}{print}\PY{p}{(}\PY{l+s+s2}{\PYZdq{}}\PY{l+s+se}{\PYZbs{}n}\PY{l+s+s2}{合并后数据预览:}\PY{l+s+s2}{\PYZdq{}}\PY{p}{)}
\PY{n+nb}{print}\PY{p}{(}\PY{n}{df\PYZus{}combined}\PY{o}{.}\PY{n}{head}\PY{p}{(}\PY{p}{)}\PY{p}{)}

\PY{c+c1}{\PYZsh{} 显示数据信息,检查列名、非空值和数据类型}
\PY{n+nb}{print}\PY{p}{(}\PY{l+s+s2}{\PYZdq{}}\PY{l+s+se}{\PYZbs{}n}\PY{l+s+s2}{合并后数据信息:}\PY{l+s+s2}{\PYZdq{}}\PY{p}{)}
\PY{n}{df\PYZus{}combined}\PY{o}{.}\PY{n}{info}\PY{p}{(}\PY{p}{)}
\end{Verbatim}
\end{tcolorbox}

    \begin{Verbatim}[commandchars=\\\{\}]

合并后数据预览:
        Key       Item Type  Publication Year  \textbackslash{}
0  4WPZ4WZA  journalArticle              2022
1  8BVI224Y  journalArticle              2021
2  XUSNP58T  journalArticle              2021
3  H4XWF2NY  journalArticle              2021
4  EKJAB2RB  journalArticle              2021

                                              Author  \textbackslash{}
0  Sun, Lei; Wang, Hua; Yu, Shanshan; Zhang, Lin;{\ldots}
1                Zhou, Lina; Ma, Xiancang; Wang, Wei
2  Jin, Xuting; Ren, Jiajia; Li, Ruohan; Gao, Ya;{\ldots}
3  Feng, Wei; Wang, Jian; Yan, Xin; Zhang, Qianqi{\ldots}
4  Shao, Can; Wang, Xiaomeng; Ma, Qingyan; Zhao, {\ldots}

                                               Title  \textbackslash{}
0  Herceptin induces ferroptosis and mitochondria{\ldots}
1  Relationship between Cognitive Performance and{\ldots}
2  Global burden of upper respiratory infections {\ldots}
3  ERK/Drp1-dependent mitochondrial fission contr{\ldots}
4  Analysis of risk factors of non-suicidal self-{\ldots}

                             Publication Title  ISBN                 ISSN  \textbackslash{}
0  International journal of molecular medicine   NaN  1791-244X 1107-3756
1               Journal of affective disorders   NaN  1573-2517 0165-0327
2                            EClinicalMedicine   NaN            2589-5370
3                           Cell proliferation   NaN  1365-2184 0960-7722
4                Annals of palliative medicine   NaN  2224-5839 2224-5820

                            DOI  Url  {\ldots} Programming Language Version System  \textbackslash{}
0        10.3892/ijmm.2021.5072  NaN  {\ldots}                  NaN     NaN    NaN
1     10.1016/j.jad.2020.12.059  NaN  {\ldots}                  NaN     NaN    NaN
2  10.1016/j.eclinm.2021.100986  NaN  {\ldots}                  NaN     NaN    NaN
3             10.1111/cpr.13048  NaN  {\ldots}                  NaN     NaN    NaN
4          10.21037/apm-21-1951  NaN  {\ldots}                  NaN     NaN    NaN

  Code  Code Number Section  Session Committee History  Legislative Body
0  NaN          NaN     NaN      NaN       NaN     NaN               NaN
1  NaN          NaN     NaN      NaN       NaN     NaN               NaN
2  NaN          NaN     NaN      NaN       NaN     NaN               NaN
3  NaN          NaN     NaN      NaN       NaN     NaN               NaN
4  NaN          NaN     NaN      NaN       NaN     NaN               NaN

[5 rows x 87 columns]

合并后数据信息:
<class 'pandas.core.frame.DataFrame'>
RangeIndex: 25105 entries, 0 to 25104
Data columns (total 87 columns):
 \#   Column                Non-Null Count  Dtype
---  ------                --------------  -----
 0   Key                   25105 non-null  object
 1   Item Type             25105 non-null  object
 2   Publication Year      25105 non-null  int64
 3   Author                25105 non-null  object
 4   Title                 25105 non-null  object
 5   Publication Title     25064 non-null  object
 6   ISBN                  0 non-null      float64
 7   ISSN                  25063 non-null  object
 8   DOI                   24969 non-null  object
 9   Url                   0 non-null      float64
 10  Abstract Note         24183 non-null  object
 11  Date                  25105 non-null  object
 12  Date Added            25105 non-null  object
 13  Date Modified         25105 non-null  object
 14  Access Date           0 non-null      float64
 15  Pages                 22971 non-null  object
 16  Num Pages             0 non-null      float64
 17  Issue                 17668 non-null  object
 18  Volume                24594 non-null  object
 19  Number Of Volumes     0 non-null      float64
 20  Journal Abbreviation  25064 non-null  object
 21  Short Title           0 non-null      float64
 22  Series                0 non-null      float64
 23  Series Number         0 non-null      float64
 24  Series Text           0 non-null      float64
 25  Series Title          0 non-null      float64
 26  Publisher             0 non-null      float64
 27  Place                 41 non-null     object
 28  Language              25105 non-null  object
 29  Rights                17986 non-null  object
 30  Type                  0 non-null      float64
 31  Archive               0 non-null      float64
 32  Archive Location      0 non-null      float64
 33  Library Catalog       0 non-null      float64
 34  Call Number           0 non-null      float64
 35  Extra                 25105 non-null  object
 36  Notes                 0 non-null      float64
 37  File Attachments      0 non-null      float64
 38  Link Attachments      0 non-null      float64
 39  Manual Tags           22640 non-null  object
 40  Automatic Tags        0 non-null      float64
 41  Editor                0 non-null      float64
 42  Series Editor         0 non-null      float64
 43  Translator            0 non-null      float64
 44  Contributor           0 non-null      float64
 45  Attorney Agent        0 non-null      float64
 46  Book Author           0 non-null      float64
 47  Cast Member           0 non-null      float64
 48  Commenter             0 non-null      float64
 49  Composer              0 non-null      float64
 50  Cosponsor             0 non-null      float64
 51  Counsel               0 non-null      float64
 52  Interviewer           0 non-null      float64
 53  Producer              0 non-null      float64
 54  Recipient             0 non-null      float64
 55  Reviewed Author       0 non-null      float64
 56  Scriptwriter          0 non-null      float64
 57  Words By              0 non-null      float64
 58  Guest                 0 non-null      float64
 59  Number                0 non-null      float64
 60  Edition               0 non-null      float64
 61  Running Time          0 non-null      float64
 62  Scale                 0 non-null      float64
 63  Medium                0 non-null      float64
 64  Artwork Size          0 non-null      float64
 65  Filing Date           0 non-null      float64
 66  Application Number    0 non-null      float64
 67  Assignee              0 non-null      float64
 68  Issuing Authority     0 non-null      float64
 69  Country               0 non-null      float64
 70  Meeting Name          0 non-null      float64
 71  Conference Name       0 non-null      float64
 72  Court                 0 non-null      float64
 73  References            0 non-null      float64
 74  Reporter              0 non-null      float64
 75  Legal Status          0 non-null      float64
 76  Priority Numbers      0 non-null      float64
 77  Programming Language  0 non-null      float64
 78  Version               0 non-null      float64
 79  System                0 non-null      float64
 80  Code                  0 non-null      float64
 81  Code Number           0 non-null      float64
 82  Section               0 non-null      float64
 83  Session               0 non-null      float64
 84  Committee             0 non-null      float64
 85  History               0 non-null      float64
 86  Legislative Body      0 non-null      float64
dtypes: float64(66), int64(1), object(20)
memory usage: 16.7+ MB
    \end{Verbatim}

    \subsection{下一步}\label{ux4e0bux4e00ux6b65}

\begin{itemize}
\tightlist
\item
  进入\texttt{data\_process.ipynb}继续下一步数据处理
\end{itemize}


    % Add a bibliography block to the postdoc
    
    
    
\end{document}
